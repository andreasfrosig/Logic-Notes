\documentclass[10pt,a4paper]{article}
\usepackage[utf8]{inputenc}
\usepackage{amsmath}
\usepackage{amsfonts}
\usepackage{amssymb}
\usepackage{fullpage}
\usepackage{array}
\begin{document}
	\subsection{Natural Deduction}
	\subsubsection{Rules}
	{
	\begin{tabular}{m{3cm} p{4cm} m{3cm} p{4cm}}
		
		Introduction rules: & meaning	& Elimination rules & meaning\\ 
		$ \left( \land I\right) \frac{A,B}{A\land B} $ & \normalsize if we have concluded A and B separately vi can conclude A and B	& $\left( \land E\right) \frac{A \land B}{A}, \frac{A \land B}{B} $ & \normalsize if we have B and A we know A must be true  \\[1cm]  
		\hline
		$ \left( \lor I\right) \frac{A}{A\lor B} , \frac{B}{A\lor B} $ & \normalsize if we have A we can introduce B & $(\lor E) \frac{A \lor B \begin{matrix}
			[A] & [B]  \\
			\vdots & \vdots  \\
			C & C
			\end{matrix}}{C} $ & \normalsize If by assuming A and B both can derive C and we know that A or B is true we can conclude C
		
		\\[1cm]
		\hline
		$(\to I) \frac{\begin{matrix}
			[A]  \\
			\vdots   \\
			B 
			\end{matrix}}{A\to B} $ & \normalsize if we by assuming A gets a B we can conclude that A implies B & $ (\to E) \frac{A,A\to B}{B} $ & if we know A an that A implies B we can conclude B \\[1cm] 
		\hline
		$ (\neg I) \frac{\begin{matrix}
			[\neg A]  \\
			\vdots   \\
			\bot 
			\end{matrix}}{\neg A} $ & \normalsize if we assume A and reaches a falsum we can conclude that A is not the case & $ (\neg E) \frac{A,\neg A}{\bot} $ & if we reach a A and $ \neg $A we can conclude a falsum\\[1cm]
		\hline
		Ex falsum quodlibet & & Reductio ad absurdum:\\ 
		$ (\bot ) \frac{\bot}{A} $ & from a falsum anything can be derived & $ (RA) \frac{\begin{matrix}
			[\neg A]  \\
			\vdots   \\
			\bot 
			\end{matrix}}{A} $ & direct opposed of $ \neg I $
	
	\end{tabular}
	\subsubsection{Syntax}
	important notes for syntax:
	\begin{itemize}
		\item $ []^x $ around assumptions where x is the number of the assumption
		\item name of the rule used at the left og the fraction line; $$ \left( \lor I\right) \frac{A}{A\lor B} $$
		\item name the number of the assumption you neutralize at the right of the fraction line:$$\frac{A}{A\lor B}x $$
		\item assumptions can be introduced when ever you want, but only removed is a rule allow it ( $ \neg I $ ) and all assumptions must be used in the end($ \to I $)
	\end{itemize}
 	
	 
\end{document}
