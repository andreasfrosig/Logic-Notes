\section{Propositional  Logic}

\subsection{Definition}

\begin{equation}
  \label{eq:1}
  \phi := p \vert \lnot \phi \vert \phi \lor \phi
\end{equation}

Where $p$ is a proposition \index{proposition}. A proposition is logical variable that can hold the values $\top$ or $\bot$.

\subparagraph{Examples on propositions}

\begin{itemize}
\item Your mom is fat!
\item The Moon is made of cheese!
\end{itemize}

\paragraph{Rules}

Where $p$ is a proposition.

\begin{equation}
  \begin{matrix}
    \lnot \lnot p & \mathbf{iff} &p \\
  \end{matrix}
\end{equation}

\paragraph{Derived Notation}

\begin{equation}
  \label{equ:notation-prop}
  \begin{matrix}
   \phi \land \phi & \mathbf{iff} & \lnot ( \lnot \phi \lor \lnot \phi ) \\
   \phi \to \phi & \mathbf{iff} & \lnot \phi \lor \phi \\ 
   \phi  \leftrightarrow \phi & \mathbf{iff}& \phi \lor \phi \\ 
  \end{matrix}
\end{equation}


\paragraph{Recap}

\begin{figure}[H]
  \centering
  \begin{tabular}{lcc}
    \textsc{name} & \textsc{in language} & \textsc{notation} \\\hline
    negation & not A & $\lnot A$ \\ \hline
    conjunction & A and B & $A \land B$ \\ \hline
    disjunction & A or B & $A \lor B$ \\ \hline
    implication & if A then B & $A \to B$ \\ \hline
    biconditional & B if and only if A & $A \leftrightarrow B$ \\ \hline

  \end{tabular}
  \caption{Description of notation}
  \label{fig:propositional}
\end{figure}

\subsection{Derived Definitions}

\paragraph{Literal}
A literal is a propositional constant or variable or its negation. 


\paragraph{Elementary disjunction or conjunction}
Is a disjunction or conjunction of one or more literals.

\subsection{Forms}

Every propositional formula is equivalent to a disjunctive normal form and to a conjunctive normal form

\paragraph{Conjunctive normal form (CNF)}

Is a logical propsitional formula where we conjunct one or more elementary disjunctions. And's outside and Or's inside.

\subparagraph{Examples}

\begin{itemize}
\item $(p \lor q) \land (\neg p \lor q)$
\end{itemize}

\paragraph{Disjunctive normal form (DNF)}

Same as CNF but have Or's outside and And's inside.

\subparagraph{Examples}

\begin{itemize}
\item $(p \land q) \lor (\neg p \land q)$
\end{itemize}

\paragraph{Algorithm to transform into CNF /DNF}

\begin{itemize}
\item Eliminate all occurrences of $\leftrightarrow$ and $\to$ using the equivalence  from \ref{equ:notation-prop}.
\item Transform to negation normal form by using the relevant equivalences. 
\end{itemize}

%%% Local Variables: 
%%% mode: latex
%%% TeX-master: "../../Notes"
%%% End: 




\subsection{Deduction Systems}

%\input{sections/natrualdeduction}



%%% Local Variables: 
%%% mode: latex
%%% TeX-master: "../../Notes"
%%% End: 
