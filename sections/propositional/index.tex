\section{Propositional  Logic}

\subsection{Definition}

\begin{equation}
  \label{eq:1}
  \phi := p \vert \lnot \phi \vert \phi \lor \phi
\end{equation}

Where $p$ is a proposition \index{proposition}. A proposition is logical variable that can hold the values $\top$ or $\bot$.

\subparagraph{Examples on propositions}

\begin{itemize}
\item Your mom is fat!
\item The Moon is made of cheese!
\end{itemize}

\paragraph{Rules}

Where $p$ is a proposition.

\begin{equation}
  \begin{matrix}
    \lnot \lnot p & \mathbf{iff} &p \\
  \end{matrix}
\end{equation}

\paragraph{Derived Notation}

\begin{equation}
  \label{equ:notation-prop}
  \begin{matrix}
   \phi \land \phi & \mathbf{iff} & \lnot ( \lnot \phi \lor \lnot \phi ) \\
   \phi \to \phi & \mathbf{iff} & \lnot \phi \lor \phi \\ 
   \phi  \leftrightarrow \phi & \mathbf{iff}& \phi \lor \phi \\ 
  \end{matrix}
\end{equation}


\paragraph{Recap}

\begin{figure}[H]
  \centering
  \begin{tabular}{lcc}
    \textsc{name} & \textsc{in language} & \textsc{notation} \\\hline
    negation & not A & $\lnot A$ \\ \hline
    conjunction & A and B & $A \land B$ \\ \hline
    disjunction & A or B & $A \lor B$ \\ \hline
    implication & if A then B & $A \to B$ \\ \hline
    biconditional & B if and only if A & $A \leftrightarrow B$ \\ \hline

  \end{tabular}
  \caption{Description of notation}
  \label{fig:propositional}
\end{figure}


\subsection{Derived Definitions}

\paragraph{Literal}
A literal is a propositional constant or variable or its negation. 


\paragraph{Elementary disjunction or conjunction}
Is a disjunction or conjunction of one or more literals.

\subsection{Forms}

Every propositional formula is equivalent to a disjunctive normal form and to a conjunctive normal form

\paragraph{Conjunctive normal form (CNF)}

Is a logical propsitional formula where we conjunct one or more elementary disjunctions. And's outside and Or's inside.

\subparagraph{Examples}

\begin{itemize}
\item $(p \lor q) \land (\neg p \lor q)$
\end{itemize}

\paragraph{Disjunctive normal form (DNF)}

Same as CNF but have Or's outside and And's inside.

\subparagraph{Examples}

\begin{itemize}
\item $(p \land q) \lor (\neg p \land q)$
\end{itemize}

\paragraph{Algorithm to transform into CNF /DNF}

\begin{itemize}
\item Eliminate all occurrences of $\leftrightarrow$ and $\to$ using the equivalence  from \ref{equ:notation-prop}.
\item Transform to negation normal form by using the relevant equivalences. 
\end{itemize}

%%% Local Variables: 
%%% mode: latex
%%% TeX-master: "../../Notes"
%%% End: 

\documentclass[10pt,a4paper]{article}
\usepackage[utf8]{inputenc}
\usepackage{amsmath}
\usepackage{amsfonts}
\usepackage{amssymb}
\usepackage{fullpage}
\usepackage{array}
\begin{document}
	\subsection{Natural Deduction}
	\subsubsection{Rules}
	{
	\begin{tabular}{m{3cm} p{4cm} m{3cm} p{4cm}}
		
		Introduction rules: & meaning	& Elimination rules & meaning\\ 
		$ \left( \land I\right) \frac{A,B}{A\land B} $ & \normalsize if we have concluded A and B separately vi can conclude A and B	& $\left( \land E\right) \frac{A \land B}{A}, \frac{A \land B}{B} $ & \normalsize if we have B and A we know A must be true  \\[1cm]  
		\hline
		$ \left( \lor I\right) \frac{A}{A\lor B} , \frac{B}{A\lor B} $ & \normalsize if we have A we can introduce B & $(\lor E) \frac{A \lor B \begin{matrix}
			[A] & [B]  \\
			\vdots & \vdots  \\
			C & C
			\end{matrix}}{C} $ & \normalsize If by assuming A and B both can derive C and we know that A or B is true we can conclude C
		
		\\[1cm]
		\hline
		$(\to I) \frac{\begin{matrix}
			[A]  \\
			\vdots   \\
			B 
			\end{matrix}}{A\to B} $ & \normalsize if we by assuming A gets a B we can conclude that A implies B & $ (\to E) \frac{A,A\to B}{B} $ & if we know A an that A implies B we can conclude B \\[1cm] 
		\hline
		$ (\neg I) \frac{\begin{matrix}
			[\neg A]  \\
			\vdots   \\
			\bot 
			\end{matrix}}{\neg A} $ & \normalsize if we assume A and reaches a falsum we can conclude that A is not the case & $ (\neg E) \frac{A,\neg A}{\bot} $ & if we reach a A and $ \neg $A we can conclude a falsum\\[1cm]
		\hline
		Ex falsum quodlibet & & Reductio ad absurdum:\\ 
		$ (\bot ) \frac{\bot}{A} $ & from a falsum anything can be derived & $ (RA) \frac{\begin{matrix}
			[\neg A]  \\
			\vdots   \\
			\bot 
			\end{matrix}}{A} $ & direct opposed of $ \neg I $
	
	\end{tabular}
	\subsubsection{Syntax}
	important notes for syntax:
	\begin{itemize}
		\item $ []^x $ around assumptions where x is the number of the assumption
		\item name of the rule used at the left og the fraction line; $$ \left( \lor I\right) \frac{A}{A\lor B} $$
		\item name the number of the assumption you neutralize at the right of the fraction line:$$\frac{A}{A\lor B}x $$
		\item assumptions can be introduced when ever you want, but only removed is a rule allow it ( $ \neg I $ ) and all assumptions must be used in the end($ \to I $)
	\end{itemize}
 	
	 
\end{document}

\subsection{Tableaux}
\subsubsection{Rules}
	\begin{tabular}{m{3cm} m{3cm} }
	Non-branching rules($\alpha$)  & Branching rules\\
	$ (\land) \begin{matrix}
	A\land B:T\\
	\downarrow\\
	A:T,B:T
	\end{matrix} $
	&$ (\land)  \begin{matrix}
	A\land B:F\\
	\swarrow \quad \searrow\\
	A:F \quad B:F
	\end{matrix}$\\
	$ (\lor F) \begin{matrix}
	A\lor B:F\\
	\downarrow\\
	A:F,B:F
	\end{matrix} $ &
	$ (\lor T) \begin{matrix}
	A\lor B:T\\
	\swarrow \quad \searrow\\
	A:T\quad B:T
	\end{matrix} $ \\
		$ (\to F) \begin{matrix}
		A\to B:F\\
		\downarrow\\
		A:T,B:F
		\end{matrix} $ &
		$ (\to T) \begin{matrix}
		A\to B:T\\
		\swarrow \quad \searrow\\
		A:F\quad B:T
		\end{matrix} $ \\
		$ (\neg T) 
		\begin{matrix}
		\neg A:T\\
		\downarrow\\
		A:F
		\end{matrix} $ &
		$ (\neg F) 
		\begin{matrix}
		\neg A:F\\
		\downarrow\\
		A:T
		\end{matrix} $
		
		 
	\end{tabular}



%%% Local Variables: 
%%% mode: latex
%%% TeX-master: "../../Notes"
%%% End: 
